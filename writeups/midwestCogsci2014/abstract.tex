\documentclass[english]{article}
\usepackage{graphicx}
\usepackage[T1]{fontenc}
\usepackage{babel}
\usepackage[group-separator={,}]{siunitx}


\title{Abstract Submission for Fourth Annual Midwestern Cognitive Science Conference}

\author{Clayton Stanley (clayton.stanley@rice.edu) \\
  Department of Psychology, 6100 Main Street \\
  Houston, TX 77005 USA 
  \and Michael D. Byrne (byrne@rice.edu) \\
  Departments of Psychology and Computer Science, 6100 Main Street \\
  Houston, TX 77005 USA \\
}

\begin{document}

\maketitle

%TC:break Abstract
\begin{abstract}
  This research explores and further validates ACT-R's base-level learning mechanism on two large-scale real-world datasets.
  We framed the process of choosing tags and hashtags on StackOverflow posts and Twitter tweets as a declarative memory retrieval problem.
  Over \num{265000} posts and 2.4 million tweets from \num{3900} total authors were collected. 
  ACT-R's base-level learning equation and the simplified version (which assumes equally-spaced retrievals for each chunk) were used to predict each author's chosen hashtags, given their previous hashtag use.
  The model performance space was explored across 22 different levels of the decay rate parameter and 48 different dataset subsets.
  The results show that for these datasets, the optimal value for the decay rate parameter for the simplified equation is 0.4, close to the standard 0.5 value. 
  When using the standard equation, the optimal decay rate increases to 0.7 and model accuracy improves from 29\% to 34\%. 
  Further, the standard equation can be implemented in a computationally efficient manner (20 ms per retrieval).
  Taken together, this research suggests that when modeling retrievals on a broader range of tasks --- or at least similar tasks with a large amount of retrievals over an extended period of time --
  it may be worthwhile to test if model accuracy improves after using the standard base-level learning equation and slightly increasing the decay rate.
  The results also highlight the importance of customizing model retrieval predictions based on each user's prior history,
  as tag retrieval accuracy is 34\% for these datasets when using only base-level learning and no contextual information.
\end{abstract}
%TC:break _main_


\end{document}


