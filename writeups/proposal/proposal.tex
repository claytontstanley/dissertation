\documentclass[man]{apa6}


\usepackage[american]{babel}
\usepackage[utf8]{inputenc}
%\usepackage[compact]{titlesec}

\usepackage{amssymb,amsmath}

\usepackage{graphicx}
\usepackage[group-separator={,}]{siunitx}
\RequirePackage[l2tabu, orthodox]{nag}
\graphicspath{{./figures/}} % Specifies the directory where pictures are stored

\usepackage{csquotes}
\usepackage[style=apa,sortcites=true,sorting=nyt,backend=biber]{biblatex}
\DeclareLanguageMapping{american}{american-apa}
\addbibresource{bibliography.bib}

\title{Psyc 601 Multivariate Statistics Project}
\shorttitle{Psyc 601 Project}

\author{Clayton Stanley}
\affiliation{Rice University}

\leftheader{Beitzel}

\abstract{
  The base StackOverflow tag-prediction model from the Fall '12 Psychometrics course was improved by incorporating the words in the body of the post and the post author's specific tagging history.
  We focused on hashtag creation as a specific human behavior, since understanding and modeling hashtag creation can lead to improved human-computer systems that can better identify users' goals and interests.
  We used multivariate logistic regression statistical techniques to guide the model building process, and found that adding the two model terms improved model accuracy from 56\% to 69\%.
  The model is a successful case showing that ACT-R's declarative memory retrieval equations scale, and are relevant to task domains that require large-scale knowledge stores.
}


\keywords{StackOverflow, Machine Learning, ACT-R, Tagging, Large-Scale Semantic Memory}

\begin{document}
\maketitle

\tableofcontents
\newpage

\section{Motivation}

TODO: Motivation

\subsection{Research Questions}

\subsubsection{Comparison Between ACT-R and Vector-based Memory Systems}

\subsubsection{Modeling Lifetime of a User's Hashtag}

\subsubsection{Single-user-focused Hashtag Modeling}

\section{Hashtag Prediction}

\subsection{Recommendation Systems}

\subsubsection{Netflix}

\subsubsection{Facebook}

\subsubsection{Twitter}

\section{ACT-R Declarative Memory Theory}

\subsection{ACT-R DM Model}

\subsection{Pointwise Mutual Information}

\section{Matrix-based Models of Semantics}

\subsection{Latent Semantic Analysis}

\subsubsection{Word Order}

\subsubsection{Comparison to PMI}

TODO: Issue of scale.

\section{Vector-based Models of Semantics}

\subsection{Addressing Word Order}

\subsection{Addressing Scalability}

\subsection{BEAGLE}

\subsection{Random Permutations}

\subsubsection{Connection to LSA}

\section{Methods}

\subsection{StackOverflow Dataset}

\subsection{Twitter Dataset}

\subsection{Analyses}

\subsubsection{Top Hashtag Prediction}

\subsubsection{Lifetime of a User's Hashtag}

\subsubsection{User-customized Hashtag Prediction}

\printbibliography

\end{document}

